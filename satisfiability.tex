\documentclass[12pt]{article}
\usepackage[english, bulgarian]{babel}
\usepackage[utf8]{inputenc}
\usepackage[T2A]{fontenc}
\usepackage{amssymb}
\usepackage{hyperref, fancyhdr, lastpage, fancyvrb, tcolorbox, titlesec}
\usepackage{array, tabularx, colortbl}
\usepackage{tikz}
\usepackage{venndiagram}
\usepackage{amsthm, bm}
\usepackage{relsize}
\usepackage{amsmath,physics}
\usepackage{mathtools}
\usepackage{subcaption}
\usepackage{theoremref}
\usepackage{circuitikz}
\usepackage{geometry}
\usepackage{stmaryrd}
\usepackage{forest}
\usepackage{cancel}
\usepackage{faktor}
\usepackage{gensymb}
\usetikzlibrary{automata, arrows, positioning, shapes}
\useforestlibrary{linguistics}

\ExplSyntaxOn
\NewDocumentCommand{\opair}{m}
 {
  \langle\mspace{2mu}
  \clist_set:Nn \l_tmpa_clist { #1 }
  \clist_use:Nn \l_tmpa_clist {,\mspace{3mu plus 1mu minus 1mu}\allowbreak}
  \mspace{2mu}\rangle
}
\ExplSyntaxOff

\hypersetup{
    colorlinks=true,
    linktoc=all,
    linkcolor=blue
}

\setlength\parindent{0pt}

\newcommand{\N}{\mathbb{N}}
\newcommand{\Z}{\mathbb{Z}}
\newcommand{\Q}{\mathbb{Q}}
\newcommand{\R}{\mathbb{R}}
\newcommand{\E}{\mathbb{E}}

\newcommand{\vars}{\operatorname{Vars}}
\newcommand{\free}{\operatorname{Free}}

\newcommand{\logand}{\; \& \;}

\newcommand{\calA}{\mathcal{A}}
\newcommand{\calS}{\mathcal{S}}
\newcommand{\calD}{\mathcal{D}}
\newcommand{\calL}{\mathcal{L}}
\newcommand{\calZ}{\mathcal{Z}}
\newcommand{\calQ}{\mathcal{Q}}
\newcommand{\calR}{\mathcal{R}}
\newcommand{\calN}{\mathcal{N}}
\newcommand{\calM}{\mathcal{M}}
\newcommand{\calF}{\mathcal{F}}
\newcommand{\calP}{\mathcal{P}}
\newcommand{\calC}{\mathcal{C}}
\newcommand{\calG}{\mathcal{G}}
\newcommand{\calB}{\mathcal{B}}

\newcommand{\dequiv}{\stackrel{\text{деф.}}{\longleftrightarrow}}

\newcommand{\db}[1]{\llbracket #1 \rrbracket}

\newtheorem*{definition}{Дефиниция}
\newtheorem{problem}{Задача}[section]
\newtheorem*{claim}{Твърдение}
\newtheorem*{property}{Свойство}
\newtheorem*{hint}{Упътване}
\theoremstyle{definition}
\newtheorem*{solution}{Решение}
\newtheorem*{notation}{Нотация}
\theoremstyle{remark}
\newtheorem*{remark}{Забележка}

\title{Задачи върху изпълнимост}
\author{Тодор Дуков}
\date{}

\begin{document}
\maketitle

\section{Стандартни структури}

\begin{problem}
Да се докаже, че следното множество от формули е изпълнимо:
\begin{align*}
       & \varphi_1 : \forall x \neg p(x, x) \; \& \; \forall x \exists y \, p(x, y)                 \\
       & \varphi_2 : \forall x \forall y \forall z (p(x, y) \; \& \; p(y, z) \Rightarrow p(x, z))   \\
       & \varphi_3 : \forall x \forall y (p(x, y) \Rightarrow \exists z (p(x, z) \; \& \; p(z, y))) \\
       & \varphi_4 : \exists x \forall y (p(x, y) \lor x \doteq y)                                  \\
       & \varphi_5 : \exists x \exists y (\neg p(x, y) \; \& \; \neg p(y, x)).
\end{align*}
\end{problem}

\begin{problem}
Да се докаже, че следното множество от формули е изпълнимо:
\begin{align*}
       & \varphi_1 : \forall x \neg p(x, x)                                                            \\
       & \varphi_2 : \forall x \forall y \forall z (p(x, y) \; \& \; p(y, z) \Rightarrow p(x, z))      \\
       & \varphi_3 : \forall x \forall y (q(x, y) \Rightarrow p(x, y))                                 \\
       & \varphi_4 : \forall x \exists y \exists z (q(x, y) \; \& \; q(x, z) \; \& \; y \not \doteq z) \\
       & \varphi_5 : \forall x \forall y (p(x, y) \Rightarrow f(x, y) \doteq x)                        \\
       & \varphi_6 : \forall x \forall y \forall z (f(x, f(y, z)) \doteq f(f(x, y), z)).
\end{align*}
\end{problem}

\begin{problem}
Да се докаже, че следното множество от формули е изпълнимо:
\begin{align*}
       & \varphi_1 : \forall x \neg p(x, x) \; \& \; \forall x \exists y \, p(x, y)                                                                                                \\
       & \varphi_2 : \forall x \forall y \forall z (p(x, y) \; \& \; p(y, z) \Rightarrow p(x, z))                                                                                  \\
       & \varphi_3 : \forall x \forall y ((q(x) \Leftrightarrow q(y)) \; \& \; p(x, y) \Rightarrow \exists z ((q(x) \Leftrightarrow \neg q(z)) \; \& \; p(x, z) \; \& \; p(z, y))) \\
       & \varphi_4 : \forall x \exists y ((q(x) \Leftrightarrow \neg q(s(x))) \; \& \; s(y) \doteq x)                                                                              \\
       & \varphi_5 : \forall x \forall y (s(x) \doteq s(y) \Rightarrow x \doteq y)                                                                                                 \\
\end{align*}
\end{problem}

\begin{problem}
Да се докаже, че следното множество от формули е изпълнимо:
\begin{align*}
       & \varphi_1 : \forall x \forall y \forall z (f(x, f(y, z)) \doteq f(f(x, y), z) \; \& \; g(x, g(y, z)) \doteq g(g(x, y), z)) \\
       & \varphi_2 : \forall x \forall y (f(x, y) \doteq f(y, x)) \; \& \; \exists x \exists y (g(x, y) \not \doteq g(y, x))        \\
       & \varphi_3 : \forall x \forall y (r(x) \; \& \; r(y) \Rightarrow r(f(x, y)) \; \& \; r(g(x, y)))                            \\
       & \varphi_4 : \forall x (r(x) \Rightarrow r(s(x)))                                                                           \\
       & \varphi_5 : \forall x (s(s(x)) \doteq s(x))                                                                                \\
       & \varphi_6 : \exists x \exists y (r(x) \; \& \; \neg r(y)).
\end{align*}
\end{problem}

\begin{problem}
Да се докаже, че следното множество от формули е изпълнимо:
\begin{align*}
       & \varphi_1 : \forall x \forall y \forall z (f(x, g(y, z)) \doteq g(f(x, y), f(x, z)))                                    \\
       & \varphi_2 : \forall x \forall y (f(x, y) \doteq f(y, x) \; \& \; g(x, y) \doteq g(y, x))                                \\
       & \varphi_3 : \forall x  (f(x, x) \doteq x \; \& \; g(x, x) \doteq x)                                                     \\
       & \varphi_4 : \forall x \forall y (f(x, y) \doteq h(g(h(x), h(y))) \; \& \; h(h(x)) \doteq x \; \& \; h(x) \not \doteq x) \\
       & \varphi_5 : \forall x (f(x, a) \doteq a \; \& \; g(x, a) \doteq x)                                                      \\
       & \varphi_6 : \forall x (g(x, b) \doteq b \; \& \; f(x, b) \doteq x)                                                      \\
       & \varphi_7 : h(a) \doteq b \; \& \; h(b) \doteq a.
\end{align*}
\end{problem}

\begin{problem}
Да се докаже, че следното множество от формули е изпълнимо:
\begin{align*}
       & \varphi_1 : \forall x \forall y \forall z (f(x, f(y, z)) \doteq f(f(x, y), z) \; \& \; g(x, g(y, z)) \doteq g(g(x, y), z))                           \\
       & \varphi_2 : \forall x \forall y (f(x, y) \doteq f(y, x)) \; \& \; \exists x \exists y (g(x, y) \not \doteq g(y, x))                                  \\
       & \varphi_3 : \forall x \forall y \forall z (g(x, f(y, z)) \doteq f(g(x, y), g(x, z)) \; \& \; g(f(x, y), z) \doteq f(g(x, z), g(y, z)))               \\
       & \varphi_4 : \forall x \exists y (f(x, y) \doteq a) \; \& \; \exists x \exists y (x \not \doteq a \; \& \; y \not \doteq a \; \& \; g(x, y) \doteq a) \\
       & \varphi_5 : \forall x (f(x, a) \doteq x \; \& \; g(x, a) \doteq a \; \& \; g(x, b) \doteq x).
\end{align*}
\end{problem}

\newpage

\section{Нестандартни структури}

\begin{problem}
Да се докаже, че следното множество от формули е изпълнимо:
\begin{align*}
       & \varphi_1 : \forall x \forall y (p(x, y) \Rightarrow \forall z (p(z, x) \lor p(y, z)))        \\
       & \varphi_2 : \forall x \forall y \forall z (p(x, y) \; \& \; p(y, z) \Rightarrow \neg p(x, z)) \\
       & \varphi_3 : \forall x \forall y (p(x, y) \Rightarrow \neg p(y, x)).
\end{align*}
\end{problem}

\begin{problem}
Да се докаже, че следното множество от формули е изпълнимо:
\begin{align*}
       & \varphi_1 : \forall x (p(x, x) \Leftrightarrow \neg \exists y \, p(x, y))                      \\
       & \varphi_2 : \forall x \forall y (p(x, y) \Leftrightarrow \exists z (p(x, z) \; \& \; p(z, y))) \\
       & \varphi_3 : \forall x \forall y (p(x, y) \Rightarrow p(f(y), f(x)))                            \\
       & \varphi_4 : \forall x \exists y (p(x, y) \; \& \; p(y, f(y)) \; \& \; p(f(y), f(x))).
\end{align*}
\end{problem}

\begin{problem}
Да се докаже, че следното множество от формули е изпълнимо:
\begin{align*}
       & \varphi_1 : \forall x (p(x, x) \; \& \; \neg r(x, x))                                                                   \\
       & \varphi_2 : \forall x \forall y \forall z (p(x, y) \; \& \; p(y, z) \Rightarrow p(x, z)))                               \\
       & \varphi_3 : \forall x \forall y (p(y, f (x)) \Leftrightarrow \exists z (r(x, z) \; \& \; f (z) \doteq y))               \\
       & \varphi_4 : \forall x \forall y \forall z (r(x, z) \; \& \; r(x, y) \; \& \; f (y) \doteq f (z) \Rightarrow y \doteq z) \\
       & \varphi_5 : \forall x \exists y \, r(y, x).
\end{align*}
\end{problem}

\begin{problem}
Да се докаже, че следното множество от формули е изпълнимо:
\begin{align*}
       & \varphi_1 : \forall x \forall y (\neg p(x, x) \; \& \; (p(x, y) \Leftrightarrow \exists z (p(x, z) \; \& \; p(z, y))))                                      \\
       & \varphi_2 : \forall x \forall y ((r(x) \Leftrightarrow r(y)) \Leftrightarrow (p(x, y) \lor p(y, x) \lor x \doteq y))                                        \\
       & \varphi_3 : \forall x \forall y (f(x) \doteq f(y) \Rightarrow x \doteq y) \; \& \; \forall x (r(x) \Leftrightarrow \neg r(f(x)))                            \\
       & \varphi_4 : \forall x \exists y \, p(x, y) \; \& \; \forall x (r(x) \Rightarrow \exists y (f(y) \doteq x))                                                  \\
       & \varphi_5 : \forall x (\neg r(x) \Rightarrow \forall y (p(x, y) \Rightarrow \exists z (p(x, z) \; \& \; p(z, y) \; \& \; \neg \exists t (f(t) \doteq z)))).
\end{align*}
\end{problem}

\newpage

\section{Теоретични задачи}

\begin{remark}
       Голяма част от задачите тук изискват знания от лекции, тоест е възможно в този момент читателя да няма апарата, който е нужен за решаването на тези задачи.
\end{remark}

\begin{problem}[Свойство на крайните модели]
Нека $\calL$ е предикатен език, съставен единствено от едноместните предикатни символи $p_1, \dots, p_n$.
Да се докаже, че за всяко множество от формули $\Gamma$ за езика $\calL$ е изпълнено, че:
\begin{center}
       $\Gamma$ има модел $\longleftrightarrow$ $\Gamma$ има краен модел.
\end{center}
Какво можем да извлечем от този резултат за разрешимостта на въпроса за изпълнимост на формули в езика $\calL$?
\end{problem}

\begin{problem}[Компактност]
Нека $\calL$ е предикатен език.
Да се докаже, че всяко множество от формули $\Gamma$ за езика $\calL$ е изпълнено, че:
\begin{center}
       $\Gamma$ е изпълнимо $\longleftrightarrow$ всяко крайно подмножество на $\Gamma$ е изпълнимо.
\end{center}
Упътване. Използвайте факта, че:
\begin{center}
       $\Gamma$ е изпълнимо $\longleftrightarrow$ $\Gamma \not \vdash_\calR \square$.
\end{center}
\end{problem}

\begin{problem}[Топологично сортиране]
Да се докаже, че всяка частична наредба може да се разшири до линейна.
\end{problem}

\begin{problem}[Лема на Kőnig]
Във всяко изброимо дърво, където всеки връх има крайно много деца, има път, който минава през всеки етаж.
\end{problem}

\begin{problem}
Нека $\calL$ е предикатен език с формално равенство.
Да се докаже, че ако едно множество от формули $\Gamma$ за езика $\calL$ има произволно големи крайни модели, то тогава $\Gamma$ има безкраен модел.
\end{problem}

\end{document}
