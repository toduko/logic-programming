\documentclass[12pt]{article}
\usepackage[english, bulgarian]{babel}
\usepackage[utf8]{inputenc}
\usepackage[T2A]{fontenc}
\usepackage{amssymb}
\usepackage{hyperref, fancyhdr, lastpage, fancyvrb, tcolorbox, titlesec}
\usepackage{array, tabularx, colortbl}
\usepackage{tikz}
\usepackage{venndiagram}
\usepackage{amsthm, bm}
\usepackage{relsize}
\usepackage{amsmath,physics}
\usepackage{mathtools}
\usepackage{subcaption}
\usepackage{scalerel}
\usepackage{theoremref}
\usepackage{circuitikz}
\usepackage{geometry}
\usepackage{stmaryrd}
\usepackage{forest}
\usepackage{cancel}
\usepackage{faktor}
\usepackage{gensymb}
\usepackage{xskak}
\usetikzlibrary{automata, arrows, positioning, shapes}
\useforestlibrary{linguistics}

\ExplSyntaxOn
\NewDocumentCommand{\opair}{m}
 {
  \langle\mspace{2mu}
  \clist_set:Nn \l_tmpa_clist { #1 }
  \clist_use:Nn \l_tmpa_clist {,\mspace{3mu plus 1mu minus 1mu}\allowbreak}
  \mspace{2mu}\rangle
}
\ExplSyntaxOff

\hypersetup{
    colorlinks=true,
    linktoc=all,
    linkcolor=blue
}

\setlength\parindent{0pt}

\newcommand{\N}{\mathbb{N}}
\newcommand{\Z}{\mathbb{Z}}
\newcommand{\Q}{\mathbb{Q}}
\newcommand{\R}{\mathbb{R}}
\newcommand{\E}{\mathbb{E}}

\newcommand{\vars}{\operatorname{Vars}}
\newcommand{\free}{\operatorname{Free}}

\newcommand{\logand}{\; \& \;}

\newcommand{\calA}{\mathcal{A}}
\newcommand{\calS}{\mathcal{S}}
\newcommand{\calD}{\mathcal{D}}
\newcommand{\calL}{\mathcal{L}}
\newcommand{\calZ}{\mathcal{Z}}
\newcommand{\calQ}{\mathcal{Q}}
\newcommand{\calR}{\mathcal{R}}
\newcommand{\calN}{\mathcal{N}}
\newcommand{\calM}{\mathcal{M}}
\newcommand{\calF}{\mathcal{F}}
\newcommand{\calP}{\mathcal{P}}
\newcommand{\calC}{\mathcal{C}}
\newcommand{\calG}{\mathcal{G}}
\newcommand{\calB}{\mathcal{B}}

\newcommand{\dequiv}{\stackrel{\text{деф.}}{\longleftrightarrow}}

\newcommand{\db}[1]{\llbracket #1 \rrbracket}

\DeclareMathOperator*{\bigand}{\scalerel*{\&}{\sum}}

\newtheorem*{definition}{Дефиниция}
\newtheorem{problem}{Задача}
\newtheorem*{claim}{Твърдение}
\newtheorem*{property}{Свойство}
\newtheorem*{hint}{Упътване}
\theoremstyle{definition}
\newtheorem*{solution}{Решение}
\newtheorem*{notation}{Нотация}
\theoremstyle{remark}
\newtheorem*{remark}{Забележка}

\title{Пролог -- 2024/2025}
\author{Тодор Дуков}
\date{}

\begin{document}
\maketitle

\begin{problem}
Нека $\calA = \opair{\{ a, b \}, \{1, \dots, n \}, 1, \delta, F}$ е детерминиран тотален краен автомат.
Представяне на автомата $\calA$ в Пролог ще наричаме термa $\mathtt{(Delta, FinalStates)}$, където $\mathtt{Delta}$ е списък, който представя графиката на $\delta$ (тоест списъкът $\mathtt{[(1, a, \delta(1, a)), (1, b, \delta(1, b)), \dots, (n, a, \delta(n, a)), (n, b, \delta(n, b))]}$) и $\mathtt{FinalStates}$ е списък с елементи измежду числата $1, \dots, n$.

Да се дефинира на Пролог предикат $\mathtt{minimise\_automaton(A, MinA)}$, който при подадено представяне $\mathtt{A}$ на автомат $\calA$ генерира в $\mathtt{MinA}$ представяне на $\calA_{\min}$, за който:
\begin{itemize}
    \item $\calA_{\min}$ е минимален автомат;
    \item $\calL(\calA) = \calL(\calA_{\min})$.
\end{itemize}
\begin{remark}
    Можем да улесним съвсем малко задачата, като дадем еднобуквена азбука.
    Тогава и представянето на $\delta$ може да бъде масив с $n$ елемента, който съдържа числа измежду $1$ и $n$.
\end{remark}
\end{problem}

\begin{problem}
Нека $G = \opair{\{ a, b \}, {1, \dots, n}, 1, R}$ е безконтекстна граматика без $\varepsilon$-правила, която има безкраен език.
Представяне на граматиката $G$ в Пролог ще наричаме всеки списък $\mathtt{Rules}$, който е представя множеството $R$ (тоест $i \rightarrow_G \alpha_1 \cdots \alpha_k$ т.с.т.к. $\mathtt{(i, [\alpha_1, \dots, \alpha_k])}$ е елемент на $\mathtt{Rules}$).

Да се дефинира на Пролог предикат $\mathtt{derive(Rules, Word)}$, който при подадено представяне $\mathtt{Rules}$ на граматика $G$ при преудовлетворяване генерира в $\mathtt{Word}$ всеки списък от вида:
\[
    \mathtt{[\alpha_1, \dots, \alpha_k]} \text{, където } \alpha_1 \cdots \alpha_k \in \calL(G).
\]
\begin{remark}
    Не е задължително граматиката да генерира безкраен език.
    Човек може да провери дали една граматика $G$ има безкраен език алгоритмично, като провери дали съществува $\alpha \in \calL(G)$, за която $p < |\alpha| < 2p$ (където $p$ е числото от Бар-Хилел лемата), и след това може да ограничи дължината на извода на думата, в случай че езика е краен, иначе генерира всички изводи и филтрира думите.
    Мисля, че този детайл е най-добре да се спести в полза на студентите.
\end{remark}
\end{problem}

\begin{problem}
Нека $\calN = \opair{\{ a, b \}, Q, S, \Delta, F}$ е недетерминиран краен автомат.
Представяне на автомата $\calN$ в Пролог ще наричаме термa $\mathtt{(Q, S, D, F)}$, където:
\begin{itemize}
    \item $\mathtt{Q}$ е списък, който представя множеството $Q$;
    \item $\mathtt{S}$ е списък, който представя множеството $S$;
    \item $\mathtt{D}$ е списък, който представя множеството $\Delta$;
    \item $\mathtt{F}$ е списък, който представя множеството $F$.
\end{itemize}
Да се дефинира на Пролог предикат $\mathtt{convert\_nfa\_to\_total\_dfa(N, A)}$, който при подадено представяне $\mathtt{N}$ на автомат $\calN$ генерира в $\mathtt{A}$ представяне на $\calA$, за който:
\begin{itemize}
    \item $\calA$ е детерминиран тотален краен автомат;
    \item $\calL(\calN) = \calL(\calA)$.
\end{itemize}
\end{problem}

\begin{problem}
Дефинираме представянето на регулярен израз $r$ над азбуката $\{ a, b \}$ в Пролог индуктивно:
\begin{itemize}
    \item представянията на $a, b, \varepsilon$ и $\varnothing$ са съответно $\mathtt{a, b, eps}$ и $\mathtt{nothing}$;
    \item представянията на $(r_1 + r_2)$ и $(r_1 \cdot r_2)$ са съответно $\mathtt{(R1 + R2)}$ и $\mathtt{(R1 * R2)}$, където $\mathtt{R1}$ и $\mathtt{R2}$ са представянията на $r_1$ и $r_2$;
    \item представянето на $r^*$ е $\mathtt{star(R)}$, където $\mathtt{R}$ е представянето на $r$.
\end{itemize}

Да се дефинира на Пролог предикат $\mathtt{regex(R)}$, който при преудовлетворяване генерира представянето на всеки регулярен израз.
\end{problem}

\begin{problem}
Представяне на едно множество $\{ a_1, \dots, a_n \}$ в Пролог е масивът $\mathtt{[a_1, \dots, a_n]}$, където $\mathtt{a_1, \dots, a_n}$ са представянията на обектите $a_1, \dots, a_n$.

Например можем да представим $\{ \varnothing, \{ \varnothing \}, \{ \varnothing, \{ \varnothing \} \} \}$ като списъка $\mathtt{[[], [[]], [[], [[]]]]}$.
Дефинираме множествата $V_n$ по рекурсия така:
\begin{align*}
     & V_0 = \varnothing       \\
     & V_{n + 1} = \calP(V_n).
\end{align*}
Да се дефинира на Пролог предикат $\mathtt{hereditarily\_finite\_set(S)}$, който при преудовлетворяване генерира в $\mathtt{S}$ представянето на всеки елемент на $\bigcup\limits_{n < \omega} V_n$.
\begin{remark}
    Тази задача може малко да се усложни, ако поискаме допълнително да се махнат ординалите, или някой друг интересен клас от множества.
\end{remark}
\end{problem}

\newpage

\begin{problem}
В Пролог ще представяме релациите като списъци от двойки.
Например $\mathtt{[(1, 2), (2, 2), (3, 1)]}$ представя релацията $\{ \opair{1, 2}, \opair{2, 2}, \opair{3, 1} \}$.

Да се дефинира на Пролог предикат $\mathtt{topo\_sort(PO, LO)}$, който при подадено представяне на частична наредба $\mathtt{PO}$ при преудовлетворяване генерира в $\mathtt{LO}$ представянето на всяка линейна наредба, която разширява представяната от $\mathtt{PO}$ наредба над същото поле.
\end{problem}

\begin{problem}
Нека $\opair{P, \leq}$ е ч.н.м.
$F \subseteq P$ ще наричаме филтър в $\opair{P, \leq}$, ако:
\begin{itemize}
    \item $F \neq \varnothing$;
    \item за всяко $x \in F$ и за всяко $y \in P$, ако $x \leq y$, то тогава $y \in F$;
    \item за всяко $x, y \in F$ съществува $z \in F$, за което $z \leq x$ и $z \leq y$.
\end{itemize}
Да се дефинира на Пролог предикат $\mathtt{gen\_filter(P, Leq, F)}$, който при подадено представяне $\mathtt{(P, Leq)}$ на някоя частична наредба $\opair{P, \leq}$ при преудовлетворяване генерира в $\mathtt{F}$ представянето на всеки един филтър в $\opair{P, \leq}$.
\begin{remark}
    Тази задача може малко да се усложни, ако поискаме да генерират ултрафилтър вместо филтър.
\end{remark}
\end{problem}

\end{document}