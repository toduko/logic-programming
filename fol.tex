\documentclass[12pt]{article}
\usepackage[english, bulgarian]{babel}
\usepackage[utf8]{inputenc}
\usepackage[T2A]{fontenc}
\usepackage{amssymb}
\usepackage{hyperref, fancyhdr, lastpage, fancyvrb, tcolorbox, titlesec}
\usepackage{array, tabularx, colortbl}
\usepackage{tikz}
\usepackage{venndiagram}
\usepackage{amsthm, bm}
\usepackage{relsize}
\usepackage{amsmath,physics}
\usepackage{mathtools}
\usepackage{subcaption}
\usepackage{theoremref}
\usepackage{circuitikz}
\usepackage{geometry}
\usepackage{stmaryrd}
\usepackage{forest}
\usepackage{cancel}
\usepackage{faktor}
\usepackage{gensymb}
\usepackage{xskak}
\usetikzlibrary{automata, arrows, positioning, shapes}
\useforestlibrary{linguistics}

\ExplSyntaxOn
\NewDocumentCommand{\opair}{m}
 {
  \langle\mspace{2mu}
  \clist_set:Nn \l_tmpa_clist { #1 }
  \clist_use:Nn \l_tmpa_clist {,\mspace{3mu plus 1mu minus 1mu}\allowbreak}
  \mspace{2mu}\rangle
}
\ExplSyntaxOff

\hypersetup{
    colorlinks=true,
    linktoc=all,
    linkcolor=blue
}

\setlength\parindent{0pt}

\newcommand{\N}{\mathbb{N}}
\newcommand{\Z}{\mathbb{Z}}
\newcommand{\Q}{\mathbb{Q}}
\newcommand{\R}{\mathbb{R}}
\newcommand{\E}{\mathbb{E}}

\newcommand{\vars}{\operatorname{Vars}}
\newcommand{\free}{\operatorname{Free}}

\newcommand{\logand}{\; \& \;}

\newcommand{\calA}{\mathcal{A}}
\newcommand{\calS}{\mathcal{S}}
\newcommand{\calD}{\mathcal{D}}
\newcommand{\calL}{\mathcal{L}}
\newcommand{\calZ}{\mathcal{Z}}
\newcommand{\calQ}{\mathcal{Q}}
\newcommand{\calR}{\mathcal{R}}
\newcommand{\calN}{\mathcal{N}}
\newcommand{\calM}{\mathcal{M}}
\newcommand{\calF}{\mathcal{F}}
\newcommand{\calP}{\mathcal{P}}
\newcommand{\calC}{\mathcal{C}}
\newcommand{\calG}{\mathcal{G}}
\newcommand{\calB}{\mathcal{B}}

\newcommand{\dequiv}{\stackrel{\text{деф.}}{\longleftrightarrow}}

\newcommand{\db}[1]{\llbracket #1 \rrbracket}

\newtheorem*{definition}{Дефиниция}
\newtheorem{problem}{Задача}[section]
\newtheorem*{claim}{Твърдение}
\newtheorem*{property}{Свойство}
\newtheorem*{hint}{Упътване}
\theoremstyle{definition}
\newtheorem*{solution}{Решение}
\newtheorem*{notation}{Нотация}

\title{Предикатно смятане от първи ред}
\author{Тодор Дуков}
\date{}

\begin{document}
\maketitle

\section{Синтаксис}

Първо фиксираме едно изброимо множество от променливи $\vars$.
\begin{notation}
Обикновено елементите на $\vars$ ще означаваме с буквите $x, y, z, t, u, v, w$ с потенциални индекси, примове и така нататък.
\end{notation}
След това фиксираме логическите символи $\exists, \neg, \lor$, помощните символи $(, )$ и \textbf{,} (символът запетая) и символ $\doteq$ за формално равенство, които не са елементи на $\vars$.

\begin{definition}
    Език на предикатното смятане от първи ред (или накратко предикатен език) ще наричаме всяко $\calL = \opair{\calC, \calF, \calP, \#}$, за което:
    \begin{itemize}
        \item $\calC$ е множество, чиито елементи ще наричаме константни символи;
        \item $\calF$ е множество, чиито елементи ще наричаме функционални символи;
        \item $\calP$ е множество, чиито елементи ще наричаме предикатни символи;
        \item $\calC, \calF$ и $\calP$ са две по две чужди;
        \item нито един от фиксираните символи не попада в $\calC, \calF$ или $\calP$, потенциално с изключение на $\doteq$, който е възможно да е елемент на $\calP$;
        \item $\# : \calF \cup \calP \rightarrow \N \setminus \{ 0 \}$ ще наричаме функця на арност, като винаги $\#(\doteq) = 2$ (ако е добре дефинирано).
    \end{itemize}
    Ако $\#(\zeta) = n$, то тогава ще наричаме $\zeta$ $n$-местен функционален/предикатен символ.
\end{definition}

\newpage

\begin{notation}
    Обикновено ще използваме:
    \begin{itemize}
        \item буквите $a, b, c, d$ за константни символи;
        \item буквите $f, g, h$ за функционални символи;
        \item буквите $p, q, r, s$ за предикатни символи,
    \end{itemize}
    като позволяваме да има индекси, примове и така нататък.
\end{notation}

\begin{definition}
    Дефинираме индуктивно понятието терм в предикатен език $\calL = \opair{\calC, \calF, \calP, \#}$:
    \begin{itemize}
        \item всеки елемент на $\calC \cup \vars$ е терм в $\calL$;
        \item ако $\tau_1, \dots, \tau_n$ са термове в $\calL$ и $f \in \calF$ има свойството $\#(f) = n$, то тогава $f(\tau_1, \dots, \tau_n)$ е терм в $\calL$.
    \end{itemize}
\end{definition}

\begin{notation}
    Термовете винаги ще бележим с буквата $\tau$ с потенциални индекси, примове и така нататък.
\end{notation}

\begin{definition}
    Дефинираме индуктивно понятието формула в предикатен език $\calL = \opair{\calC, \calF, \calP, \#}$:
    \begin{itemize}
        \item за всеки $n$-местен предикатен символ $p$ и термове $\tau_1, \dots, \tau_n$ е изпълнено, че $p(\tau_1, \dots, \tau_k)$ е (атомарна) формула в $\calL$;
        \item ако $\varphi$ и $\psi$ са формули в $\calL$, то тогава $\neg \varphi$ и $(\varphi \lor \psi)$ са формули в $\calL$;
        \item ако $x$ е променлива и $\varphi$ е формула в $\calL$, то тогава $\exists x \varphi$ е формула в $\calL$.
    \end{itemize}
\end{definition}

\begin{notation}
    Формулите винаги ще бележим с буквите $\varphi, \psi, \chi$ с потенциални индекси, примове и така нататък.
\end{notation}

\begin{definition}
    За формула или терм $\alpha$ с $\vars[\alpha]$ ще бележим множеството от всички променливи, които се срещат в $\alpha$.
\end{definition}

\begin{definition}
    Всеки терм $\tau$, за който $\vars[\tau] = \varnothing$, се нарича затворен.
\end{definition}

\begin{notation}
    Под $\tau[x_1, \dots, x_n]$ ще имаме предвид ``$\tau$ е терм с променливи измежду $x_1, \dots, x_n$'', тоест $\vars[\tau] \subseteq \{ x_1, \dots, x_n \}$.
\end{notation}

\newpage

\begin{definition}
    Дефинираме множеството от свободни променливи $\free[\varphi]$ за формула $\varphi$ в предикатен език $\calL$ с индукция относно построението на $\varphi$:
    \begin{itemize}
        \item $\free[p(\tau_1, \dots, \tau_n)] = \bigcup\limits_{i = 1}^n \vars[\tau_i]$;
        \item $\free[(\varphi \lor \psi)] = \free[\varphi] \cup \free[\psi]$;
        \item $\free[\neg \varphi] = \free[\varphi]$;
        \item $\free[\exists x \varphi] = \free[\varphi] \setminus \{ x \}$.
    \end{itemize}
\end{definition}

\begin{notation}
    За двуместни функционални/предикатни символи $\zeta$ въвеждаме записа $\tau_1 \zeta \tau_2$ като съкращение на $\zeta(\tau_1, \tau_2)$.
\end{notation}

\begin{definition}
    Всяка формула $\varphi$, за която $\free[\varphi] = \varnothing$, се нарича затворена.
\end{definition}

\begin{notation}
    Под $\varphi[x_1, \dots, x_n]$ ще имаме предвид ``$\varphi$ е формула със свободни променливи измежду $x_1, \dots, x_n$'', тоест $\free[\varphi] \subseteq \{ x_1, \dots, x_n \}$.
\end{notation}

\begin{notation}
    Нека фиксираме новите символи $\&, \Rightarrow, \Leftrightarrow, \forall$.
    Въвеждаме следните съкращения:
    \begin{itemize}
        \item $(\varphi \logand \psi)$ е съкращение за $\neg (\neg \varphi \lor \neg \psi)$;
        \item $(\varphi \Rightarrow \psi)$ е съкращение за $(\neg \varphi \lor \psi)$;
        \item $(\varphi \Leftrightarrow \psi)$ е съкращение за $((\varphi \Rightarrow \psi) \logand (\psi \Rightarrow \varphi))$;
        \item $\forall x$ е съкращение за $\neg \exists x \neg$;
        \item $\varphi_1 \, \sigma \, \dots \, \sigma \, \varphi_n$ е съкращение за $(\varphi_1 \, \sigma \, (\varphi_2 \, \sigma \, (\dots (\varphi_{n - 1} \, \sigma \, \varphi_n) \dots )))$ за $\sigma \in \{ \lor, \& \}$;
        \item няма да пишем най-външните скоби, ако формулата започва с такива.
    \end{itemize}
\end{notation}

\newpage

\section{Семантика}

\begin{definition}
    Структура за предикатен език $\calL = \opair{\calC, \calF, \calP, \#}$ ще наричаме всяко $\calA = \opair{A, C, F, P}$, за което:
    \begin{itemize}
        \item $A \neq \varnothing$ се нарича универсум или носител на $\calA$;
        \item $C \subseteq A$ и за всяко $c \in \calC$ има $c^\calA \in C$;
        \item за всяко $f \in \calF$ с $\#(f) = n$ има $f^\calA \in F$, като $f^\calA : A^n \rightarrow A$;
        \item за всяко $p \in \calP$ с $\#(p) = n$ има $p^\calA \in F$, като $p^\calA \subseteq A^n$ и в частност $\doteq^\calA$ винаги играе ролята на идентитет (при език с формално равенство).
    \end{itemize}
\end{definition}

\begin{notation}
    Обикновено структурите ще бележим с ръкописни латински букви $\calA, \calB, \dots, \calZ$ с потенциални индекси, примове и така нататък.
\end{notation}

\begin{notation}
    Вместо да използваме записа $\calA = \opair{A, C, F, P}$ ще изреждаме всички символи, тоест ще пишем $\calA = \opair{A, c_0^\calA, c_1^\calA, \dots, f_0^\calA, f_1^\calA, \dots, p_0^\calA, p_1^\calA, \dots}$.
\end{notation}

\begin{notation}
    Универсум на структурата $\calA$ ще бележим с $|\calA|$.
\end{notation}

\begin{notation}
    За да няма конфликти между обектния език и метаезика ще използваме $f[x]$ вместо $f(x)$, когато искаме образът на $x$ през $f$.
\end{notation}

\begin{definition}
    Нека $\calL$ е предикатен език, $\tau[x_1, \dots, x_n]$, $\calA$ е структура за $\calL$ и $a_1, \dots, a_n \in |A|$.
    С индукция относно построението на $\tau$ дефинираме $\tau^\calA \db{a_1, \dots, a_n}$:
    \begin{itemize}
        \item $c^\calA \db{a_1, \dots, a_n} = c^\calA$;
        \item $x_i^\calA \db{a_1, \dots, a_n} = a_i$;
        \item $f(\tau_1, \dots, \tau_k)^\calA \db{a_1, \dots, a_n} = f^\calA[\tau_1^\calA \db{a_1, \dots, a_n}, \dots, \tau_k^\calA \db{a_1, \dots, a_n}]$.
    \end{itemize}
\end{definition}

\begin{definition}
    Нека $\calL$ е предикатен език, $\varphi[x_1, \dots, x_n]$, $\calA$ е структура за $\calL$ и $a_1, \dots, a_n \in |A|$.
    Дефинираме $\calA \models \varphi \db{a_1, \dots, a_n}$ индуктивно:
    \begin{itemize}
        \item $\calA \models p(\tau_1, \dots, \tau_k) \db{a_1, \dots, a_n} \dequiv \opair{\tau_1 \db{a_1, \dots, a_n}, \dots, \tau_k \db{a_1, \dots, a_n}} \in p^\calA$;
        \item $\calA \models \neg \varphi \db{a_1, \dots, a_n} \dequiv \calA \not \models \varphi \db{a_1, \dots, a_n}$ (не е вярно, че $\calA \models \varphi \db{a_1, \dots, a_n}$);
        \item $\calA \models (\varphi \lor \psi) \db{a_1, \dots, a_n} \dequiv \calA \models \varphi \db{a_1, \dots, a_n} \text{ или }\calA \models \psi \db{a_1, \dots, a_n}$;
        \item $\calA \models \exists x_i \varphi \db{a_1, \dots, a_n} \dequiv \text{има } a \in |\calA| \text{, за което } \calA \models \varphi \db{a_1, \dots, a_{i - 1}, a, a_{i + 1}, \dots, a_n}$.
    \end{itemize}
\end{definition}

\begin{definition}
    Ако за всяко $a_1, \dots, a_n \in |\calA|$ е изпълнено, че $\calA \models \varphi \db{a_1, \dots, a_n}$, тогава ще пишем $\calA \models \varphi$.
\end{definition}

\newpage

\begin{property}
    Нека $\calL$ е предикатен език, $\varphi[x_1, \dots, x_n]$, $\calA$ е структура за $\calL$ и $a_1, \dots, a_n \in |A|$.
    Тогава следните са в сила:
    \begin{itemize}
        \item $\calA \models (\varphi \logand \psi) \db{a_1, \dots, a_n} \longleftrightarrow \calA \models \varphi \db{a_1, \dots, a_n} \text{ и } \calA \models \psi \db{a_1, \dots, a_n}$;
        \item $\calA \models (\varphi \Rightarrow \psi) \db{a_1, \dots, a_n} \longleftrightarrow \text{ ако }\calA \models \varphi \db{a_1, \dots, a_n} \text{, то } \calA \models \psi \db{a_1, \dots, a_n}$;
        \item $\calA \models (\varphi \Leftrightarrow \psi) \db{a_1, \dots, a_n} \longleftrightarrow \calA \models \varphi \db{a_1, \dots, a_n} \text{ т.с.т.к. } \calA \models \psi \db{a_1, \dots, a_n}$;
        \item $\calA \models \forall x_i \varphi \db{a_1, \dots, a_n} \longleftrightarrow \text{за всяко } a \in |\calA| \text{ имаме } \calA \models \varphi \db{a_1, \dots, a_{i - 1}, a, a_{i + 1}, \dots, a_n}$.
    \end{itemize}
\end{property}

\begin{definition}
    Нека $\calA$ е структура за предикатния език $\calL$.
    Казваме, че едно множество $X \subseteq |\calA|^n$ е определимо в езика $\calL$, ако има формула $\varphi$ в $\calL$, за която:
    \[
    \opair{a_1, \dots, a_n} \in X \longleftrightarrow \calA \models \varphi \db{a_1, \dots, a_n}.
    \]
\end{definition}

\begin{definition}
    Нека $\Gamma$ е множество от формули в предикатния език $\calL$ и $\calA$ е структура за $\calL$.
    Казваме, че $\calA$ e модел на $\Gamma$ и пишем $\calA \models \Gamma$, ако за всяко $\varphi \in \Gamma$ е изпълнено, че $\calA \models \varphi$.
\end{definition}

\begin{definition}
    Едно множество от формули наричаме изпълнимо, ако за него съществува модел.
\end{definition}

\begin{definition}
    Нека $\calA$ и $\calB$ са структури за предикатния език $\calL$.
    Функцията $h : |\calA| \rightarrow |\calB|$ наричаме изоморфизъм, ако:
    \begin{itemize}
        \item $h$ е биекция;
        \item $h[c^\calA] = c^\calB$ за всеки константен символ $c$;
        \item $h[f^\calA[a_1, \dots, a_n]] = f^\calB[h[a_1], \dots, h[a_n]]$ за всеки функционален символ $f$ и $a_1, \dots, a_n \in |\calA|$;
        \item $\opair{a_1, \dots, a_n} \in p^\calA \longleftrightarrow \opair{h[a_1], \dots, h[a_n]} \in p^\calB$ за всеки предикатен символ $p$ и $a_1, \dots, a_n \in |\calA|$.
    \end{itemize}
    Ако $\calA = \calB$, то тогава $h$ се нарича и автоморфизъм.
\end{definition}

\begin{property}
    Нека $\calA$ и $\calB$ са структури за предикатния език $\calL$ и $h$ е изоморфизъм между тях.
    Тогава за всеки терм $\tau[x_1, \dots, x_n]$, за всяка формула $\varphi[x_1, \dots, x_n]$ и $a_1, \dots, a_n \in |\calA|$:
    \[
        h[\tau^\calA\db{a_1, \dots, a_n}] = \tau^\calB\db{h[a_1], \dots, h[a_n]} \text{ и } \calA \models \varphi \db{a_1, \dots, a_n} \longleftrightarrow \calB \models \varphi \db{h[a_1], \dots, h[a_n]}.
    \]
\end{property}

\begin{property}
    Нека $\calA$ е структура за предикатния език $\calL$, $h$ е автоморфизъм (от $\calA$ към себе си) и $X \subseteq |\calA|^n$.
    Тогава ако $X$ е определимо, то:
    \[
        \{ \opair{h[a_1], \dots, h[a_n]} \mid \opair{a_1, \dots, a_n} \in X \} = X.
    \]
\end{property}

\end{document}
